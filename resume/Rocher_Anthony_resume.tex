%%%%%%%%%%%%%%%%%
% This is an sample CV template created using altacv.cls
% (v1.1.3, 30 April 2017) written by LianTze Lim (liantze@gmail.com). Now compiles with pdfLaTeX, XeLaTeX and LuaLaTeX.
% 
%% It may be distributed and/or modified under the
%% conditions of the LaTeX Project Public License, either version 1.3
%% of this license or (at your option) any later version.
%% The latest version of this license is in
%%    http://www.latex-project.org/lppl.txt
%% and version 1.3 or later is part of all distributions of LaTeX
%% version 2003/12/01 or later.
%%%%%%%%%%%%%%%%

%% If you need to pass whatever options to xcolor
\PassOptionsToPackage{dvipsnames}{xcolor}

%% If you are using \orcid or academicons
%% icons, make sure you have the academicons 
%% option here, and compile with XeLaTeX
%% or LuaLaTeX.
% \documentclass[10pt,a4paper,academicons]{altacv}

%% Use the "normalphoto" option if you want a normal photo instead of cropped to a circle
% \documentclass[10pt,a4paper,normalphoto]{altacv}

\documentclass[11pt, a4paper]{altacv_VB_academic}

%% AltaCV uses the fontawesome and academicon fonts
%% and packages. 
%% See texdoc.net/pkg/fontawecome and http://texdoc.net/pkg/academicons for full list of symbols.
%% 
%% Compile with LuaLaTeX for best results. If you
%% want to use XeLaTeX, you may need to install
%% Academicons.ttf in your operating system's font 
%% folder.


% Change the page layout if you need to
\geometry{left=2cm,right=10cm,marginparwidth=6.8cm,marginparsep=1.2cm,top=1.5cm,bottom=1.5cm,footskip=2\baselineskip}
%\geometry{left=1cm,right=9cm,marginparwidth=6.8cm,marginparsep=1.2cm,top=1.25cm,bottom=1.25cm,footskip=2\baselineskip}
% Change the font if you want to.

% If using pdflatex:
\usepackage[utf8]{inputenc}
\usepackage[T1]{fontenc}
%\RequirePackage{newcent}
%\renewcommand{\familydefault}{\sfdefault}         % to set the default font; use '\sfdefault' for the default sans serif font, '\rmdefault' for the default roman one, or any tex font name
\usepackage[default]{lato}

\usepackage{multicol}

% If using xelatex or lualatex:
% \setmainfont{Lato}

% Change the colours if you want to
%Bordeaux 992843
\definecolor{DBlue}{HTML}{032A97}
\definecolor{Bordeaux}{HTML}{490F0A}
\definecolor{DarkGrey}{HTML}{424242}
\definecolor{LightGrey}{HTML}{666666}
\colorlet{heading}{black}
\colorlet{accent}{DarkGrey}
\colorlet{emphasis}{DarkGrey}
\colorlet{body}{LightGrey}

\usepackage{hyperref}
\hypersetup{
    colorlinks=true,
    linkcolor=DarkGrey,
    filecolor=LightGrey,      
    urlcolor=DarkGrey,
    pdfpagemode=FullScreen,
    }

% Change the bullets for itemize and rating marker
% for \cvskill if you want to
\renewcommand{\itemmarker}{{\small-}}
\renewcommand{\ratingmarker}{\faCircle}

\begin{document}
\name{Anthony Rocher}
\personalinfo{%
  % Not all of these are required!
  % You can add your own with \printinfo{symbol}{detail}
  \email{\href{mailto:anthony.rocher@ens-lyon.fr}{anthony.rocher@ens-lyon.fr}}
 % \phone{+33 6 13 38 18 18}
 \homepage{\href{https://anthonyrocher.github.io/}{anthonyrocher.github.io}}
 % \mailaddress{}
 %\github{github.com/anthonyrocher}
  \location{Lyon, FR}
   
%  \twitter{@twitterhandle}
%  \linkedin{linkedin.com/in/yourid}
  %\github{github.com/anthonyrocher/}
  %% You MUST add the academicons option to \documentclass, then compile with LuaLaTeX or XeLaTeX, if you want to use \orcid or other academicons commands.
%   \orcid{orcid.org/0000-0000-0000-0000}
}

%% Make the header extend all the way to the right, if you want. 
\begin{fullwidth}
\makecvheader

%% Provide the file name containing the sidebar contents as an optional parameter to \cvsection.
%% You can always just use \marginpar{...} if you do
%% not need to align the top of the contents to any
%% \cvsection title in the "main" bar.

%\cvonelinelist{\textnormal{\textbf{Placement Director}}}{Kiki Pop-Eleches, cp2124@columbia.edu, +1 (212) 854 4476}

%\cvonelinelist{\textnormal{\textbf{Placement Assistant}}}{Tomara Aldrich, tsa2110@columbia.edu, +1(212)-854-9785}

\smallskip

\cvsection{Current Position}

\cvevent{\href{https://www.ens-lyon.fr/en/studies/academic-programs/masters-2022-2026/advanced-economics}{Master in Advanced Economics}}{Ecole Normale Supérieure Lyon (ENS Lyon)}{2024-Present}{}

%%%%%%%%%%%%%%%%%%%%%%%%%%%%%%%

\cvsection{Education}

\cvevent{Pre-master in Economics}{Ecole Normale Supérieure Lyon (ENS Lyon)}{2023-2024}{Lyon}

\cvevent{Preparatory class in social sciences}{Lycées Blaise Pascal, Jacques Amyot, Michel Montaigne}{2019-2023}{Clermont-Ferrand, Melun, Bordeaux}

%\cvevent{Baccalauréat}{Lycée Léonard de Vinci}{2019}{Monistrol-sur-Loire}

%%%%%%%%%%%%%%%%%%%%%%%%%%%%%%%

\cvsection{Research Interests}

\textbf{Urban and Real Estates Economics, Environmental Economics, Public Economics}
\smallskip

%%%%%%%%%%%%%%%%%%%%%%%%%%%%%%%

%\cvsection{References}
%
%\vspace{-10pt}
%\begin{multicols}{3}
%
%\cvref{\href{https://jeffreyshrader.com/}{Jeffrey Shrader}}{Columbia University\\ SIPA}{420 West 118th Street,\\New York, NY 10027, US}{+1 (212) 851-9443}{jgs2103@columbia.edu}
%
%\cvref{\href{https://www.tse-fr.eu/people/sylvain-chabe-ferret}{Sylvain Chabé-Ferret}}{Toulouse School of Economics (TSE)}{1 esplanade de l’Université\\
%31000 Toulouse, France}{+33 (0)5 61 12 88 28}{\small sylvain.chabe-ferret@tse-fr.eu}
%
%\cvref{\href{https://econ.columbia.edu/econpeople/suresh-naidu/}{Suresh Naidu}}{Columbia University\\ Dept. of Economics, SIPA}{420 West 118th Street,\\New York, NY 10027, US}{+1 (212) 854-0027}{sn2430@columbia.edu}
%
%%\cvref{Geoffrey Heal}{Columbia University\\ Business School}{665 West 130th Street,\\New York, NY 10027, US}{+1 (212) 854-6459}{gmh1@gsb.columbia.edu}
%
%\end{multicols}

%%%%%%%%%%%%%%%%%%%%%%%%%%%%%%%

\cvsection{Master thesis}

\cvworkingpaper{Seismic Hazard Mapping In Japan: effects on land and real estate prices}{}{}{}{What is the cost of being exposed to natural disasters? Focusing on earthquakes in Japan, my master’s thesis investigates the effects of risk mapping on land and real estate prices. Seismic mapping has evolved continuously since the 2000s, and new faults have been discovered. My hypothesis is that this has had a negative effect on land and houses exposed to these risks, reducing their value while controlling for other factors. I contribute to the literature on climate risk exposure and its economic consequences.}

%%%%%%%%%%%%%%%%%%%%%%%%%%%%%%%

\cvsection{Research Experience}

\cvevent{Research Intern in Information Economics}{CERGIC, with \textit{Elisa Mougin} and \textit{Camille Urvoy}}{April -- July 2025}{}

\cvevent{Senior thesis}{ENS Lyon, supervised by \textit{Sophie Hatte}, grade: A+}{September -- July 2024}{}

%%%%%%%%%%%%%%%%%%%%%%%%%%%%%%%

\cvsection{Selected projects}

\cvworkingpaper{Machine Learning project}{}{}{}{Using a disease-symptoms dataset, we want to predict the risk of catching a tumor given individual's symptoms. To perform it, we program using python a bagging tree models.}

\cvworkingpaper{Senior thesis}{}{}{}{This paper addresses the problem of persistent food waste in school canteens. We design a scalable test–control study to apply a nudge aimed at reducing food waste in school settings. First, we develop a protocol for a pilot study. Then, we propose an extended protocol for a large-scale experiment. Our main contribution to the research field is the introduction of an interactive nudge. It takes the form of a revised version of the “Climate Fresk,” involving researchers, school canteen staff, teachers, and children aged 11 to 15. We expect this intervention to significantly reduce food waste over the five-week experimental period.}

%%%%%%%%%%%%%%%%%%%%%%%%%%%%%%%

\cvsection{Teaching Experience}

%\cvsubsection{Current Teaching at ENS Lyon}

\cveventshort{Examiner at Lycée du Parc}{2025-Present}

%%%%%

%\cvsubsection{Teaching Assistant at Columbia University}

\cveventshort{Tutor: Statistics, Econometrics (Undergrad)}{Winter 2025, Winter 2026}
\cveventshort{Tutor: Macroeconomics (Undergrad)}{Fall 2024}

%%%%%%%%%%%%%%%%%%%%%%%%%%%%%%%

%\cvsection{Teaching Interests}

%Environmental Economics, Causal Inference, Principles of Economics

%Method Courses: R for Social Sciences, Simulations for Regression Analysis and Data Visualization

%%%%%%%%%%%%%%%%%%%%%%%%%%%%%%%

%%%%%%%%%%%%%%%%%%%%%%%%%%%%%%%%%%%%%%%%%%%%

\cvsection{Other information}

\cvonelinelist{Languages}{French (\textit{native}), English (\textit{fluent}), Japanese (\textit{conversational})}

\cvonelinelist{Software skills}{\cvtag{R}, \cvtag{Stata}, \cvtag{Python}, \cvtag{\LaTeX}, \cvtag{Git}, \cvtag{Matlab}}

\cvonelinelist{Citizenship}{French}
%
%\cvskill{French}{5}
%
%\smallskip
%
%\cvskill{English}{4}
%
%\smallskip
%
%\cvskill{Spanish}{3}

\end{fullwidth}

%% If the NEXT page doesn't start with a \cvsection but you'd
%% still like to add a sidebar, then use this command on THIS
%% page to add it. The optional argument lets you pull up the 
%% sidebar a bit so that it looks aligned with the top of the
%% main column.
% \addnextpagesidebar[-1ex]{page3sidebar}

\end{document}